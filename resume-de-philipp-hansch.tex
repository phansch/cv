\documentclass[12pt]{article}

%%%%%%% Pagestyle stuff %%%%%%%%%%%%%%%%%
 \usepackage{fancyhdr}                 %%
 \usepackage[paperheight=11in,%        %%
             paperwidth=8in,%          %%
             outer=1in,%               %%
             inner=1in,%               %%
             bottom=.7in,%             %%
             top=.1 in,%                %%
             includeheadfoot]{geometry}%%
%%%%%%% End Pagestyle stuff %%%%%%%%%%%%%

\usepackage[utf8]{inputenc}
\usepackage[ngerman]{babel}
\usepackage{hyperref}

\renewcommand{\familydefault}{\sfdefault}

\hypersetup{colorlinks=true,urlcolor=blue}
\usepackage[usenames,dvipsnames]{color}


\begin{document}

\thispagestyle{empty}

%--Header---------------------------------------


{\Huge Philipp \textbf{Hansch}}

{\footnotesize
  \textcolor{Gray}{
    \uppercase{Tangermünderstrasse 118}
    •
    \uppercase{12627, Berlin}
  }

  \textcolor{Gray}{
    0151 578 00 976
	•
    \uppercase{\href{mailto:desk@phansch.net}{desk@phansch.net} }
    •
    \uppercase{\href{http://github.com/phansch}{github.com/phansch}}
    •
    \uppercase{\href{http://portfolio.phansch.net/\#folio-mid}{Portfolio}}
  }
}

 % Reset font settings
\vspace{0.5cm}
%--Body---------------------------------------

Erfahrung in Software Entwicklung, zuletzt in Ruby on Rails, einschließlich

\begin{itemize}
  \setlength{\itemsep}{0.1cm}
  \setlength{\parskip}{0.1cm}
  \item Game development in Lua mit Love2D sowie C\# und XNA
  \item Desktop development in Java und C\#, mit Datenbankanschluss an Oracle und SQL auf Windows und Linux
\end{itemize}
\vspace{0.5cm}

%--Experience---------------------------------------
{\Large Erfahrung}
\begin{itemize}
  \setlength{\itemsep}{0cm}
  \setlength{\parskip}{0cm}

  \item[] \emph{Winter 2013/2014}\hfill Teilnahme in Hackership
  \item[] Lernen von Ruby on Rails in einer Selbstlerner Umgebung. Entwicklung einer Webanwendung die das Zehnfingersystem lehrt.
\end{itemize}

\begin{itemize}
  \setlength{\itemsep}{0cm}
  \setlength{\parskip}{0cm}

  \item[] \emph{Dezember 2013}\hfill Programmieranfänger Coach beim RailsGirls Berlin Hackday
  \item[] Programmieranfänger\_innen helfen die Grundlagen des Programmierens zu verstehen. Erstellung einer einfachen Rails Anwendung und beantworten von dabei aufkommenden Fragen.
\end{itemize}

\begin{itemize}
  \setlength{\itemsep}{0cm}
  \setlength{\parskip}{0cm}

  \item[] \emph{Herbst 2010}\hfill Programmier Trainee bei Modus Consult AG
  \item[] Design, Implementierung und Dokumentierung einer Beispielanwendung basierend auf Microsoft Dynamics NAV
\end{itemize}

\vspace{0.5cm}
%--Projects---------------------------------------
{\Large Projekte}

\begin{itemize}
  \setlength{\itemsep}{0cm}
  \setlength{\parskip}{0cm}
  \item[] \emph{PiDefender} \hfill \href{https://github.com/phansch/PiDefender}{Quellcode auf GitHub}

  \item[] PiDefender ist ein top-down 2D space-shooter geschrieben in Lua und Love2D.
\end{itemize}


\begin{itemize}
  \setlength{\itemsep}{0cm}
  \setlength{\parskip}{0cm}
  \item[] \emph{Jou} \hfill \href{https://github.com/phansch/jou}{Quellcode auf GitHub}

  \item[] Jou ist ein einfaches Command-Line Tool das das erstellen eines Tagebuches unterstützt.
\end{itemize}

Mehr Projekte können unter \href{http://phansch.net/portfolio/#folio-overview}{portfolio.phansch.net} gefunden werden.

\vspace{0.5cm}
%--Education---------------------------------------
{\Large Bildung}

\begin{itemize}
  \setlength{\itemsep}{0cm}
  \setlength{\parskip}{0.1cm}
  \item[] \emph{Herbst 2011 - Herbst 2013}\hfill HTW Berlin
  \item[] Verfolgen eines Bachelors in Angewandter Informatik
\end{itemize}

\begin{itemize}
  \setlength{\itemsep}{0cm}
  \setlength{\parskip}{0.1cm}
  \item[] \emph{Herbst 2007 - Frühling 2011}\hfill Carl Miele Berufskolleg für Technik des Kreises Gütersloh
  \item[] Staatlich geprüfter Informationstechnischer Assistent
\end{itemize}

\begin{itemize}
  \setlength{\itemsep}{0cm}
  \setlength{\parskip}{0.1cm}
  \item[] \emph{Herbst 2001 - Sommer 2007}\hfill Osterrath-Realschule
  \item[] Fachoberschulreife
\end{itemize}

\newpage

\vspace{0.5cm}
%--Language exposure----------------------------------
{\Large Programmierkenntnisse}
\begin{itemize}
  \setlength{\itemsep}{0cm}
  \setlength{\parskip}{0cm}
  \item[] Erfahren in: C\#, Ruby
  \item[] Vertraut mit: Lua, Java, LaTeX, SQL, PHP
  \item[] Wenig Erfahrung: Python, C, C++, Javascript
\end{itemize}

\vspace{0.5cm}
%--Tools----------------------------------------------
{\Large Tools}
\begin{itemize}
  \setlength{\itemsep}{0cm}
  \setlength{\parskip}{0cm}
  \item[] Erfahrung mit git, Sublime Text, vim, Visual Studio, NetBeans, Linux and Windows
\end{itemize}

\vspace{0.5cm}
%--Interests----------------------------------------------
{\Large Interessen}
\begin{itemize}
  \setlength{\itemsep}{0cm}
  \setlength{\parskip}{0cm}
  \item[] Blogging, open-source, game development und Astronomie.
\end{itemize}


\end{document}
